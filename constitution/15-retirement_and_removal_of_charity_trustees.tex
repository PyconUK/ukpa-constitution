%!TEX root = constitution.tex
\section{Retirement and Removal of Charity Trustees}\label{sec:retirement_removal}

    \subsection{}\label{sec:retirement}
    A charity trustee ceases to hold office if he or she:
    \begin{enumerate}
        \item retires by notifying \shortname{} in writing (but only if enough charity trustees will remain in office when the notice of resignation takes effect to form a quorum for meetings);
        \item is absent without the permission of the charity trustees from all their meetings held within a period of six months and the trustees resolve that his or her office be vacated;
        \item dies;
        \item in the written opinion, given to \shortname{}, of a registered medical practitioner treating that person, has become physically or mentally incapable of acting as a trustee and may remain so for more than three months;
        \item is removed by the members of \shortname{} in accordance with clause~\ref{sec:removal} or
        \item\label{item:trustee_disqualification} is disqualified from acting as a charity trustee by virtue of section 178--180 of the Charities Act 2011 (or any statutory re-enactment or modification of that provision).
    \end{enumerate}

    \subsection{}\label{sec:removal}
    A charity trustee shall be removed from office if a resolution to remove that trustee is proposed at a general meeting of the members called for that purpose and properly convened in accordance with clause~\ref{sec:general_meetings}, and the resolution is passed by a two-thirds majority of votes cast at the meeting.

    \subsection{}
    A resolution to remove a charity trustee in accordance with this clause shall not take effect unless the individual concerned has been given at least 14 clear days' notice in writing that the resolution is to be proposed, specifying the circumstances alleged to justify removal from office, and has been given a reasonable opportunity of making oral and/or written representations to the members of \shortname{}.
